% LaTeX resume using res.cls
\documentclass[margin]{res}
\setlength{\textwidth}{5.1in} % set width of text portion
\renewcommand{\labelitemi}{$\vcenter{\hbox{\tiny$\bullet$}}$} % reduce size of bullet in lists

% link formatting
\usepackage[hidelinks]{hyperref}
% for older versions of hyperref
\hypersetup{
	colorlinks=false,
	pdfborder={0 0 0},
}

% justify text without hyphenation
\tolerance=1
\emergencystretch=\maxdimen
\hyphenpenalty=10000
\hbadness=10000

% typeset C++ and C# properly
\newcommand{\PLUS}{\nolinebreak\hspace{-.05em}\raisebox{.4ex}{\tiny\bf+}}
\newcommand{\CC}{C\PLUS{}\PLUS{}}
\newcommand{\CS}{C\nolinebreak\hspace{-.05em}\raisebox{.4ex}{\scriptsize\bf \#}}

\begin{document}

% Center the name over the entire width of resume:
\moveleft.5\hoffset\centerline{\fontsize{16}{19}\bf Daniel Cheng}
% Draw a horizontal line the whole width of resume:
\moveleft\hoffset\vbox{\hrule width\resumewidth height 1pt}\smallskip
% contact info here
% Again, the contact info must be centered over entire width of resume:
% \moveleft.5\hoffset\centerline{2500 University Avenue}
% \moveleft.5\hoffset\centerline{Austin, TX 78705}
\moveleft.5\hoffset\centerline{\href{mailto:danielcheng@utexas.edu}{danielcheng@utexas.edu}}
\moveleft.5\hoffset\centerline{\href{tel:12142127050}{(214) 212-7050}}
\moveleft.5\hoffset\centerline{\href{https://github.com/qwerpi}{https://github.com/qwerpi}}
% \moveleft.5\hoffset\centerline{\href{http://www.cs.utexas.edu/users/danielhc}{http://www.cs.utexas.edu/users/danielhc}}
\moveleft.5\hoffset\centerline{\href{https://www.linkedin.com/in/danielcheng15}{https://www.linkedin.com/in/danielcheng15}}

% reduce vertical space
\vspace{-4mm}

\begin{resume}
 
\section{EDUCATION}
	Bachelor of Science, Computer Science \hfill 2012 -- 2015\\
	The University of Texas at Austin\\
	Turing Scholars Honors Degree\\
	GPA: 3.8 / 4.0\\
	Expected Graduation Date: May 2015

\vspace{-1mm}

\section{WORK EXPERIENCE}
	Software Developer Intern \hfill Summer 2013, Summer 2014 \\
	CodePartners LLC
	\begin{itemize} \itemsep -1pt %reduce space between items
	\item Created frontend and backend of web-based toolkit that provides clients with an intuitive and efficient way to manage data
	\item Developed toolkit using the Intacct API, ASP.NET MVC 4, \CS{}, jQuery, HTML5, CSS, and WebSockets
	\item Created cloud-based development and testing VMs using Windows Azure
	\item Created logo animations and video introductions using Blender
	\end{itemize}
 
	Student IT Technician \hfill Fall 2012 -- present \\
	Division of Housing and Food Service, The University of Texas at Austin
	\begin{itemize} \itemsep -2pt %reduce space between items
	\item Provide front-line tech support to administrative users in offices, residence halls, kitchens, facilities shops, and University apartments
	\item Install and maintain computer hardware, software, printers, and other peripheral devices in a 200\PLUS{} workstation business computing environment
	\end{itemize}

\section{RESEARCH}
	BLIS (BLAS Library Instantiation Software) for GPUs \hfill Fall 2014 -- present
	\begin{itemize} \itemsep -2pt
	\item Automatic code generation for dense linear algebra libraries targeting GPUs using OpenCL
	\item Explore parallelization methods for scientific computing libraries
	\end{itemize}

\section{COURSE PROJECTS}

	Artificial Intelligence Honors: Pacman in Python \hfill Spring 2014
	\begin{itemize} \itemsep -2pt
	\item Implemented concepts in the A.I. field as pacman-playing intelligent agents
	\item Implemented algorithms for search (bfs, dfs, A*), multi-agent planning (minimax), reinforcement learning (Q-learning), tracking (particle filters), and classification (Bayes, perceptron)
	\end{itemize}

	Programming for Performance \hfill Fall 2013
	\begin{itemize} \itemsep -2pt
	\item Implemented vectorized, parallelized, and blocked algorithms for fast matrix operations and data visualization
	\item Implemented efficient graph representations and algorithms
	\end{itemize}

	Autonomous Intelligent Robotics: Freshman Research Initiative \hfill Spring 2013
	\begin{itemize} \itemsep -2pt
	\item Programmed Segway robots to follow objects, detect and avoid obstacles, and plan paths in \CC{} using ROS and OpenCV
	\item Read and interpreted data from cameras, depth sensors, and laser range sensors
	\end{itemize}

\section{SKILLS}
	Programming Languages/Tools: Java, C, \CC{}, \CS{}, ASP.NET MVC, JavaScript, jQuery, Python, Visual Studio, \LaTeX\\
	Written/Spoken Languages: English (native), Mandarin Chinese (fluent), Classical Latin (beginner)\\

\end{resume}
\end{document}
